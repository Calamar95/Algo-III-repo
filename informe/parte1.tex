\subsection{Idea general del problema}
Se ha decidido conectar telegráficamente todas las estaciones de un sistema férreo que recorre el país en abanico con origen en la capital (el kilómetro 0). Se nos ofrece cierta cantidad de kilometros de cable para conectar la ciudades de cada ramal. Al ser escaso el presupuesto, se busca lograr conectar la mayor cantidad de ciudades con los metros asignados, sin hacer cortes en el cable. \\

Se nos propone resolver cuántas ciudades se pueden conectar para cada ramal, con una complejidad de O(n), siendo n la cantidad de estaciones en cada ramal.\\

Para ello se nos brinda un archivo de entrada, el cual tiene para cada ramal dos líneas: la primera contiene un entero con los kilómetros de cable dedicados al ramal y la segunda los kilometrajes de las estaciones en el ramal sin considerar el 0. Luego de ejecutar nuestro algoritmo, la salida del mismo debe contener, para cada ramal de la entrada, una línea con la cantidad de ciudades conectables encontradas.\\

Un ejemplo de archivo de entrada puede ser, (extracto del archivo Tp1Ej1.in):\\
6 \\
6 8 12 15 \\
35 \\
8 14 20 40 45 54 60 67 74 89 99 \\
100 \\
35 87 141 163 183 252 288 314 356 387 \\
90 \\
6 8 16 19 28 32 37 45 52 60 69 78 82 \\

El mismo indica, en su primer línea que para el ramal 1 tenemos 6km de cable, y en su segunda línea que dicho ramal contiene (además de la capital, implícita, en el kilómetro 0) una estación en el kilómetro 6, otra en el 8, otra en el 12 y la última en el kilómetro 15. Luego para el ramal 2, tenemos 35 kilómetros de cable, y estaciones en los kilómetros: 8 14 20 40 45 54 60 67 74 89 y 99. Así sucesivamente para el resto de los ramales.\\

El archivo de salida luego de ejecutarse nuestro algoritmo deberá ser de la siguiente pinta, (extracto del archivo Tp1Ej1.out):\\
3 \\
6 \\
4 \\
14 \\

Este último archivo indica la cantidad de ciudades que se pueden conectar para cada ramal. En el caso del ramal 1, para el cual se tienen 6km de cable disponibles, y contiene ciudades en los kilómetros: 0 6 8 12 15 vemos que la solución debería ser que se pueden conectar como máximo 3 ciudades, a continuación explicaremos cómo se deduce esto.\\

Si conectamos la capital con la ciudad del kilómetro 6, al tener sólo 6km de cable, nuestra solución sería que pudimos conectar sólo 2 ciudades. Pero como debemos maximizar esta cantidad, podemos ver que si en vez de conectar a la capital con la primer estación del ramal, conectamos la ciudad del kilómetro 6, con su siguiente y con la del kilómetro 12, entonces como entre el kilómetro 6 y el 8 hay una diferencia de 2kms y entre el 8 y el 12 una diferencia de 4kms, vemos que la máxima cantidad de estaciones conectadas con 6km de cable para el ramal 1, es 3. Lo mismo para todos los ramales.\\


\subsection{Explicación y pseudocódigo}



\subsection{Deducción de la cota de complejidad temporal}

\subsection{Demostración formal}
\subsection{Experimentaciones}